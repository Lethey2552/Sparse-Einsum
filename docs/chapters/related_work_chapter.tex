% IMPORTANT: Try using a array based approach for tensor BMM instead of
% an unordered map. The array should be sorted and then contracted based
% on the indices. The unordered map is slow for large quantities of data
% because of the hashing that has to take place.

\noindent
Compared to the well-established methods for Einsum with dense tensors, Einstein summation with
sparse tensors has received relatively little attention in the scientific community.
Due to various tensor operations that can be expressed using Einsum notation, the
underlying algorithms need to be able to handle many distinct computations. Here we
introduce multiple approaches and ideas, contributing to the field of sparse Einsum.
\\
\\
Recent developments in integrating machine learning and linear algebra routines into
databases have gained significant attention
~\cite{Machine_Learning_LinA_and_More, du2020inmachinelearningdatabasereimaginingdeep,
       data_management_in_machine_learning, deepdive}.
One such approach is the translation of sparse Einsum problems into SQL queries
\cite{sql_einsum}. The authors introduce four mapping rules and a decomposition scheme
in which large Einsum operations are split into multiple smaller Einsum operations.
In contrast to SQL-based approaches, the TACO compiler can translate known sparse linear
algebra and tensor operations into optimized code directly~\cite{taco}. While this produces
optimized code for predefined problems with trivial contraction paths, it faces limitations
in handling dynamic problems that are not known at compile time. TACO does not calculate
an efficient contraction path, nor does it allow for the application of previously computed
contraction paths. As a result other methods, capable of using optimized contraction paths
outperform TACO, especially for large tensor expressions involving thousands of higher order tensors.
Gray J. developed an Einsum function that calculates tensor expressions via a batch matrix
multiplication (BMM) approach~\cite{jcmgray}. This method allows for the computation of pairwise
tensor expressions by mapping them to BMMs, using summation over indices, transposition
and reshaping of the tensors. A BMM approach for evaluating Einstein summation expressions
is also employed by Torch within its tensor library Aten. Sparse, a library designed for
operations on sparse tensors, implements an Einsum function~\cite{sparse}. However, when used
alone, Sparse struggles with large tensor expressions due to its limitations in handling a high
number of different indices. This limitation can be overcome by using Sparse as a backend for
opt\_einsum, a package that optimizes tensor contraction orders. Sparse utilizes Numba
~\cite{lam2015numba} to accelerate calculations; Numba is a just-in-time compiler that generates
machine code from Python syntax.