Einstein summation notation is a powerful and compact notation used in mathematics
and physics to represent sums over tensor indices. It was introduced by Albert
Einstein in the early 20th century as a means to simplify expressions in the
theory of relativity. \cite{einstein1916} The notation is both elegant and efficient,
making it a valuable tool in various fields such as theoretical physics, computational
mathematics, and data science.
\\\\
The fundamental operation in Einstein summation notation, often referred to simply
as "Einsum," is the Einstein summation. This operation allows for the concise expression
of various tensor operations, including element-wise multiplication, dot products,
outer products, and matrix multiplications. The computational efficiency and
expressiveness of Einsum have led to its adoption in numerous applications,
ranging from machine learning to scientific computing.
\\\\
In many practical applications, especially in machine learning and scientific computing,
the data involved is often sparse. Sparse matrices are matrices in which most of
the elements are zero. Handling sparse matrices efficiently requires specialized
algorithms and data structures to avoid unnecessary computations and to save memory.
Traditional libraries like NumPy \cite{numpy} and other major artificial intelligence
frameworks \cite{tensorflow, pytorch} typically support Einstein summation
for dense matrices, but not for sparse matrices. The only known library that aims to
support Einsum operations on sparse tensors is Sparse \cite{sparse}. However, like NumPy,
Sparse only allows for the use of capital and small letters of the Latin alphabet as
indices, which limits the number of dimensions tensors can have. Furthermore,
real Einstein summation problems often inculde expressions with hundreds or even thousands
of higher order tensors. In order to express the aforementioned operations we require a
large set of unique symbols. Thus, our approaches are capable of handling all
symbols in the UTF-8 encoding.
\\\\
This thesis explores the implementation and performance of Einstein summation
across different computing paradigms, with a particular focus on sparse tensors.
Specifically, it focuses on comparing two distinct implementations to multiple libraries:
%
\begin{itemize}
      \item SQL-based Implementation:
            This implementation is based on the algorithm presented in "Efficient and
            Portable Einstein Summation in SQL" by Blacher et al \cite{sql_einsum}.
            It constructs SQL queries dynamically using Python. While SQL is
            traditionally used for database operations, this approach demonstrates
            the versatility of SQL in performing tensor operations.
      \item C++ Implementation: The second implementation is written in C++, with multiple
            versions ranging from naive to optimized approaches. The C++ implementations
            aim to explore the performance trade-offs between simplicity and optimization,
            offering insights into how different coding strategies affect computational efficiency.
\end{itemize}
%
% Write about some of the results of the paper. I.e. we were able to outperform
% this and that library and we accept more different indices as input.
%
By comparing these implementations, this thesis aims to provide a comprehensive
analysis of the performance and scalability of sparse Einstein summation in various
computing environments. The SQL-based implementation serves as a baseline,
showcasing the potential of database query languages for tensor operations.
Furthermore, the C++ implementations demonstrate the impact of low-level optimizations
on computational performance. Comparing these against the sparse library Sparse and
highly performance-tuned dense libraries like PyTorch provides insights into different
use cases and helps identify the optimal tool for various tasks. The code for the tools
can be found on Github https://github.com/Lethey2552/Sparse-Einsum.
\\
Through this comparative study, we seek to identify the strengths and weaknesses
of each approach, providing guidelines for selecting the appropriate
method based on specific use cases and computational requirements. This work
contributes to the broader understanding of tensor operations and their efficient
implementation, offering practical insights for researchers and practitioners
in fields that rely heavily on tensor computations.