In this chapter, we discuss potential improvements to our C++ algorithm that could
improve both the memory footprint and execution speed.

\section{Iterative Approach for Batch Matrix Multiplication and Index Reordering}
The current C++ implementation uses unordered maps to aggregate values from COO tensors.
This is an inefficient way and can be slow for big tensors because it involves the
maintenance of unordered maps. One such improvement can be using an iterative approach
for the sparse batch matrix multiplication. Here we could guarantee the indices for the batch
matrix multiplication are ordered as
\begin{equation*}
    bik, bjk \rightarrow bij.
\end{equation*}
By processing indices from left to right, we can ensure that we handle each unique index
only once, thus avoiding redundant computations and making better use of memory and
computational resources. Because the contraction path delivers an unambiguous order for
tensor operations, reordering indices to process tensors iteratively could grant high
performance savings. Index reordering will enable the algorithm to skip over portions
of tensors that have already been processed, hence avoiding useless computations. This
approach relies on the fact that, after processing a part of the tensor, it need not be visited
again, and thus it traverses and processes tensor elements in a more efficient way. Not
only does this decrease computational complexity, but it also avoids adding vast quantities
of elements to a hash map.

\section{Precomputation of Diagonal Indices and Summed Indices}
The existing algorithm processes the diagonal and summed indices right before the actual pairwise 
tensor contractions. Often, this preprocessing is inefficient in
that it contains redundant computations for indices, which can sometimes be handled far more
effectively before the main phase of contraction. One way of handling this inefficiency could
be to traverse the contraction path, that essentially forms a binary tree structure and compute
the operations given by the diagonal indices and indices to be summed over beforehand. 
\\
\\
\makebox[\linewidth][s]{For example, say we have the following Einsum expression based on three tensors:}
\begin{equation*}
    ifbk, cjhb, ckkah \rightarrow bij
\end{equation*}
Now, we evaluate it via a contraction path that indicates to compute the expression
\begin{equation*}
    \vspace{-1mm}
    cjhb, ckkah \rightarrow bjk,
\end{equation*}
\makebox[\linewidth][s]{followed by another contraction that requires the result of the previous expression}
\begin{equation*}
    ifbk, bjk \rightarrow bij.
\end{equation*}
Figure \ref{fig:traverse_tree} shows how we can traverse the binary tree from the root and
check which indices the corresponding parent requires. The diagonals and summations over indices
can be computed and the remaining indices can be reordered to fit the BMM.
\\

1.
\begin{figure}[H]
    \centering
    \begin{tikzpicture}[
            level 1/.style={sibling distance=4cm},
            level 2/.style={sibling distance=3cm},
            level 3/.style={sibling distance=2cm},
            every node/.append style={text centered},
            edge from parent path={(\tikzparentnode) -- (\tikzchildnode)}
        ]

        % Root
        \node {bij}
        child {node {ifbk}}
        child {node {bjk}
                child {node {cjhb}}
                child {node {ckkah}}
            };

    \end{tikzpicture}
\end{figure}
2.
\begin{figure}[H]
    \centering
    \begin{tikzpicture}[
            level 1/.style={sibling distance=4cm},
            level 2/.style={sibling distance=3cm},
            level 3/.style={sibling distance=2cm},
            every node/.append style={text centered},
            edge from parent path={(\tikzparentnode) -- (\tikzchildnode)}
        ]

        % Root
        \node {bij}
        child {node {\st{f}bik}}
        child {node {bkj}
                child {node {cjhb}}
                child {node {ckkah}}
            };

    \end{tikzpicture}
\end{figure}
3.
\begin{figure}[H]
    \centering
    \begin{tikzpicture}[
            level 1/.style={sibling distance=4cm},
            level 2/.style={sibling distance=3cm},
            level 3/.style={sibling distance=2cm},
            every node/.append style={text centered},
            edge from parent path={(\tikzparentnode) -- (\tikzchildnode)}
        ]

        % Root
        \node {bij}
        child {node {\st{f}bik}}
        child {node {bkj}
                child {node {\st{c}bjh}}
                child {node {\st{ack}hk}}
            };

    \end{tikzpicture}
    \caption{An example for the removal of diagonal indices and indices to be summed over, as well as
        the permutation of indices by traversing the binary tree given by the contraction path.}
    \label{fig:traverse_tree}
\end{figure}

\noindent
Such preprocessing makes it possible for the algorithm to perform fewer operations in total.
When the main computation starts, the aforementioned indices that can be removed can be
processed in the first pass. Hence, extra steps are not required. By rearranging indices
such that only required operations are made during the contraction phase, the algorithm can
reduce the overall processing time along with the computational overhead.

\section{Memory Footprint Optimization via Multi Index Encoding}
One of the possible bottlenecks in our latest implementation is that it consumes a lot of
memory while storing and tracking tensor indices in COO format. The current practice is to
store each index separately, thus making it expensive in terms of memory usage for large
tensors and high-order contractions. Extra performance gains could be attained by compressing
a number of indices into a single index through a multi-index encoding. For example, four
16-bit indices may be combined into a single 64-bit integer. Such compression would reduce
the memory footprint and could further improve computational speed by allowing bit-wise
operations on the combined index. This would complicate the tracking of the indices, but the
improvement of memory efficiency and the likely following performance gains, due to better
memory access patterns, could be substantial.