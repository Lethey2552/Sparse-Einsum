\section{Tensors}
Tensors are algebraic objects and a fundamental concept in mathematics, physics
and computer science. They extend the idea of scalars, vectors and matrices to
higher dimensions. In essence, a tensor is a multi-dimensional array with an abitrary
number of dimensions.\\
Each dimension of a tensor is represented by an index with its own range.
The number of indices is commonly referred to as the tensor's "rank" or "order."
The size of a tensor is determined by the product of the maximum values of each
index's range.\\
For example, consider a tensor $T$ with indices $i,j,k$ and corresponding ranges\\
$i \in \{1,2\},\ j \in \{1,2,3,4,5,6\}$ and $k \in \{1,2,3,4\}$. The size of
tensor $T$ is calculated as follows: $2 \cdot 6 \cdot 4 = 48$. This means tensor
$T$ has a total of $48$ elements.

\begin{figure}[H]
    \label{fig_matrix_tensor_network}
    \centering
    \begin{tikzpicture}[
            node/.style={circle, draw=black, fill=white, thick, minimum size=7mm},
            node_inv/.style={circle, draw=white, fill=white, minimum size=7mm},
        ]
        %Nodes
        \node[node_inv]  (I_1_1)                      {};
        \node[node]      (A_1)       [right=of I_1_1] {A};
        \node[node_inv]  (I_1_2)     [above=of A_1]   {};

        \node[node_inv]  (I_2_1)     [right=of A_1]   {};
        \node[node]      (A_2)       [right=of I_2_1] {A};
        \node[node_inv]  (I_2_2)     [above=of A_2]   {};

        \node[node_inv]  (I_3_1)     [right=of A_2]   {};
        \node[node]      (A_3)       [right=of I_3_1] {A};
        \node[node_inv]  (I_3_2)     [above=of A_3]   {};
        \node[node]      (B_3)       [right=of A_3]   {B};
        \node[node_inv]  (I_3_3)     [above=of B_3]   {};
        \node[node_inv]  (I_3_4)     [right=of B_3]   {};

        %Lines
        \draw[-] (A_1.west) -- (I_1_1.east) node [above, fill=white, opacity=.0, text opacity=1] {i};
        \draw[-] (A_1.north) -- (I_1_2.south) node [right, fill=white, opacity=.0, text opacity=1] {j};

        \draw[-] (A_2.west) -- (I_2_1.east) node [above, fill=white, opacity=.0, text opacity=1] {i};
        \draw[-] (A_2.north) -- (I_2_2.south) node [right, fill=white, opacity=.0, text opacity=1] {j};
        \draw[-] (A_2.east) -- (I_3_1.west) node [above, fill=white, opacity=.0, text opacity=1] {k};

        \draw[-] (A_3.west) -- (I_3_1.east) node [above, fill=white, opacity=.0, text opacity=1] {i};
        \draw[-] (A_3.north) -- (I_3_2.south) node [right, fill=white, opacity=.0, text opacity=1] {j};
        \draw[-] (A_3.east) -- (B_3.west) node [midway, above, fill=white, opacity=.0, text opacity=1] {k};
        \draw[-] (B_3.north) -- (I_3_3.south) node [right, fill=white, opacity=.0, text opacity=1] {m};
        \draw[-] (B_3.east) -- (I_3_4.west) node [above, fill=white, opacity=.0, text opacity=1] {n};
    \end{tikzpicture}
    \caption{A matrix, a tensor and a tensor network visualized as a graph. Each index is represented
        by an edge, shared indices in a tensor network are edges between nodes.}
\end{figure}

\section{Einstein Notation}
In 1916, Albert Einstein introduced the so called Einstein notation, also known as
Einstein summation convention or Einstein summation notation, for the sake of
representing tensor expressions in a concise manner. As an example, the contraction of tensors
$A \in \mathbb{R}^{I \times J \times K}$ and $B \in \mathbb{R}^{K \times M \times N}$
in figure \ref{fig_matrix_tensor_network},

\[C_{ijmn} = \sum_{k}A_{ijk} \cdot B_{kmn}\]
\noindent
\\
can be simplified by making the assumption that pairs of repeated indices in the expression
are to be summed over.
\newpage
\noindent
Consequently, the contraction can be rewritten as:

\[C_{ijmn} = A_{ijk} \cdot B_{kmn}\]
\noindent
\\
To expant upon the expressive power of the original Einstein notation, modern Einstein
notation was introduced. This notation is used by most linear algebra and machine
learning libraries that provide Einstein summation notation. Modern Einstein notation
explicitly states the indices for the output tensor, enabling further operations like
transpositions and traces.\\
In modern Einstein notation, the expression from our previous example would
be written as:

\[A_{ijk}B_{kmn} \rightarrow C_{ijmn}\]
\noindent
\\
When using common Einstein summation APIs, tensor operations are encoded by using the
indices of the tensors in a format string and the data itself. The format string for the
above operation would come down to:

\[ijk,kmn \rightarrow ijmn\]
\noindent
\\
In Modern Einstein notation, indices that are not mentioned in the output are to be
summed over. For the sake of simplicity, we will from now on refer to Einstein summation
as Einsum, and we will use the original, the modern notation or just the format
string, depending on the context.