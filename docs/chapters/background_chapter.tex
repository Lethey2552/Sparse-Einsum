The following chapter serves to introduce the necessary background for tensors,
Einstein notation and Einstein summation. Furthermore, we will provide various 
examples for operations that can be expressed using Einstein notation. Given the 
considerable overlap in topics, we will build on related literature [5], adapting 
and expanding it to meet our specific research requirements.

\section{Tensors}
Tensors are algebraic objects and a fundamental concept in mathematics, physics
and computer science. They extend the idea of scalars, vectors and matrices to
higher dimensions. In essence, a tensor is a multi-dimensional array with an arbitrary 
number of dimensions. Each dimension of a tensor is represented by an index with its 
own range. The number of indices is commonly referred to as the tensor's ``rank" or 
``order." The size of a tensor is determined by the product of the maximum values of 
each index's range.\\
For example, consider a tensor $T$ with indices $i,j,k$ and corresponding ranges
$i \in \{1,2\},\ j \in \{1,2,3,4,5,6\}$ and $k \in \{1,2,3,4\}$. The size of
tensor $T$ is calculated as follows: $2 \cdot 6 \cdot 4 = 48$. This means tensor
$T$ has a total of $48$ elements. An example of a matrix A with indices i, j and a tensor 
A with indices i, j, k, both represented as a graph, can be seen in Figure 
\ref{fig:tensor:vec_mat_ten}.

\begin{figure}[H]
    \label{fig:tensor:vec_mat_ten}
    \centering
    \begin{tikzpicture}[
            node/.style={circle, draw=black, fill=white, thick, minimum size=7mm},
            node_inv/.style={circle, draw=white, fill=white, minimum size=7mm},
        ]
        %Nodes
        \node[node_inv]  (I_1_1)                      {};
        \node[node]      (A_1)       [right=of I_1_1] {A};
        \node[node_inv]  (I_1_2)     [above=of A_1]   {};

        \node[node_inv]  (I_2_1)     [right=of A_1]   {};
        \node[node]      (A_2)       [right=of I_2_1] {A};
        \node[node_inv]  (I_2_2)     [above=of A_2]   {};

        \node[node_inv]  (I_3_1)     [right=of A_2]   {};
        \node[node]      (A_3)       [right=of I_3_1] {A};
        \node[node_inv]  (I_3_2)     [above=of A_3]   {};
        \node[node]      (B_3)       [right=of A_3]   {B};
        \node[node_inv]  (I_3_3)     [above=of B_3]   {};
        \node[node_inv]  (I_3_4)     [right=of B_3]   {};

        %Lines
        \draw[-] (A_1.west) -- (I_1_1.east) node [above, fill=white, opacity=.0, text opacity=1] {i};
        \draw[-] (A_1.north) -- (I_1_2.south) node [right, fill=white, opacity=.0, text opacity=1] {j};

        \draw[-] (A_2.west) -- (I_2_1.east) node [above, fill=white, opacity=.0, text opacity=1] {i};
        \draw[-] (A_2.north) -- (I_2_2.south) node [right, fill=white, opacity=.0, text opacity=1] {j};
        \draw[-] (A_2.east) -- (I_3_1.west) node [above, fill=white, opacity=.0, text opacity=1] {k};

        \draw[-] (A_3.west) -- (I_3_1.east) node [above, fill=white, opacity=.0, text opacity=1] {i};
        \draw[-] (A_3.north) -- (I_3_2.south) node [right, fill=white, opacity=.0, text opacity=1] {j};
        \draw[-] (A_3.east) -- (B_3.west) node [midway, above, fill=white, opacity=.0, text opacity=1] {k};
        \draw[-] (B_3.north) -- (I_3_3.south) node [right, fill=white, opacity=.0, text opacity=1] {m};
        \draw[-] (B_3.east) -- (I_3_4.west) node [above, fill=white, opacity=.0, text opacity=1] {n};
    \end{tikzpicture}
    \caption{A matrix, a tensor and a tensor network visualized as a graph. Each index is represented
        by an edge. Shared indices in a tensor network are edges between nodes.}
\end{figure}

\noindent
In this work, a tensor is simply a multidimensional array containing data of a
primitive type. We differentiate between dense and sparse tensors.

\paragraph{Dense Tensors.}
Dense tensors have a significant number of non-zero entries. However, there is no
exact threshold which determines whether a tensor is dense or sparse.

\paragraph{Sparse Tensors.}
In Sparse tensors most values are zero. They can greatly profit from specialized
formats. For our tensor $T \in \mathbb{R}^{I \times J \times K}$ in dense format
we need to save $I \cdot J \cdot K$ values no matter whether they are zero or not.
Now consider that, if the vast majority of $T$'s values are zero, we could only save
the coordinates of the non-zero values, that is the index of the value for each
dimension. This is what we call the coordinate (COO) format. A sparse tensor could
be reduced to the COO format as follows:

\begin{equation*}
    \begin{bmatrix}
        0 & 1 & 0 & 0  \\
        0 & 0 & 4 & 0  \\
        5 & 0 & 0 & 10 \\
        0 & 0 & 0 & 0
    \end{bmatrix}
    \quad
    \Rightarrow
    \quad
    \begin{bmatrix}
        0 & 1 & 1  \\
        1 & 2 & 4  \\
        2 & 0 & 5  \\
        2 & 3 & 10
    \end{bmatrix}
\end{equation*}
%
Each row of the COO representation encodes a single value of the tensor with each
column holding the position of the value for the corresponding dimension and the last
column giving the actual value. This can be done for an arbitrary number of dimensions
by simply adding more columns for their respective coordinates.

\section{Einstein Notation and Einstein Summation}
In 1916, Albert Einstein introduced the so called Einstein notation, also known as
Einstein summation convention or Einstein summation notation, for the sake of
representing tensor expressions in a concise manner. As an example, the contraction of tensors
$A \in \mathbb{R}^{I \times J \times K}$ and $B \in \mathbb{R}^{K \times M \times N}$
in Figure \ref{fig:tensor:vec_mat_ten},

\[C_{ijmn} = \sum_{k}A_{ijk} \cdot B_{kmn}\]
\noindent
\\
can be simplified by making the assumption that pairs of repeated indices in the expression
are to be summed over. Consequently, the contraction can be rewritten as:

\[C_{ijmn} = A_{ijk} \cdot B_{kmn}\]
\noindent
\\
To expand upon the expressive power of the original Einstein notation, modern Einstein
notation was introduced. This notation is used by most linear algebra and machine
learning libraries supporting Einstein summation, that is, the evaluation of the actual
tensor expressions. Modern Einstein notation explicitly states the indices for the
output tensor, enabling further operations like transpositions, traces or summation
over non shared indices. In modern Einstein notation, the expression from the previous
example would be written as:

\[A_{ijk}B_{kmn} \rightarrow C_{ijmn}\]
\noindent
\\
When using common Einstein summation APIs, tensor operations are encoded by using the
indices of the tensors in a format string and the data itself.
\newpage
\noindent
The format string for the above operation would come down to:

\[ijk,kmn \rightarrow ijmn\]
\noindent
\\
In Modern Einstein notation, indices that are not mentioned in the output are to be
summed over. For the sake of simplicity, we will from now on refer to Einstein summation
as Einsum, and we will use the original, the modern notation or just the format
string, depending on the context.

\section{Operations with Einsum}
Einsum is a powerful tool for performing various tensor operations. Table \ref{tab:einsum:ops} 
shows some common operations that can be performed using Einsum.

\begin{table}[hbp]
    \caption{Example operations with Einsum.}
    \label{tab:einsum:ops}
    \centering
    \def\arraystretch{1.1}
    \begin{tabular}{lcc}
        \toprule
        \textbf{Operation}                 & \textbf{Formula}                                    & \textbf{Format string}         \\
        \midrule
        Dot Product                        & $c = \sum_{i} a_{i} b_{i}$                          & i,i $\rightarrow$              \\
        Sum Over Axes                      & $b_{j} = \sum_{i} A_{ij}$                           & ij $\rightarrow$ j             \\
        Outer Product                      & $C_{ij} = a_{i} b_{j}$                              & i,j $\rightarrow$ ij           \\
        Matrix Multiplication              & $C_{ij} = \sum_{k} A_{ik} B_{kj}$                   & ik,kj $\rightarrow$ ij         \\
        Batch Matrix Multiplication        & $C_{bij} = \sum_{k} A_{bik} B_{bkj}$                & bik,bkj $\rightarrow$ bij      \\
        Tucker Decomposition \cite{tucker} & $T_{ijk} = \sum_{pqr} D_{pqr} A_{ip} B_{jq} C_{kr}$ & pqr,ip,jq,kr $\rightarrow$ ijk \\
        \bottomrule
    \end{tabular}
    %}
\end{table}

\noindent
These examples illustrate the versatility of Einsum in performing a wide range of 
tensor operations using a concise and readable notation, expressed as a format string.
Note that while the examples provided are relatively simple, real-world Einstein
summation problems may include thousands of tensors.

\section{Tensor Contractions}
\label{sec:tensor:contractions}
Tensor contraction is the process of reducing one or multiple tensor's orders by summing 
over pairs of matching indices. Tensor networks where more than two tensors share an index
are called tensor hypernetworks.

\begin{figure}
    \centering
    \begin{tikzpicture}[
            node/.style={circle, draw=black, fill=white, thick, minimum size=7mm},
            node_inv/.style={circle, draw=white, fill=white, minimum size=7mm},
        ]
        %Nodes

        \node[node_inv]  (I_1)                      {};
        \node[node]      (A)       [right=of I_1] {A};
        \node[node_inv]  (I_2)     [above=of A]   {};
        \node[node]      (B)       [right=of A]   {B};
        \node[node_inv]  (I_3)     [above=of B]   {};
        \node[node_inv]  (I_4)     [right=of B]   {};
        
        % Add a coordinate for the T-junction
        \coordinate (T) at ($(A)!0.5!(B)$);
        \node[node] (C) [below=of T] {C};
        \node[node_inv] (I_5) [below=of C] {};

        % Draw the T-junction edge
        \draw[-] (T) -- (C);

        %Lines
        \draw[-] (A.west) -- (I_1.east) node [above, fill=white, opacity=.0, text opacity=1] {i};
        \draw[-] (A.north) -- (I_2.south) node [right, fill=white, opacity=.0, text opacity=1] {j};
        \draw[-] (A.east) -- (B.west) node [midway, above, fill=white, opacity=.0, text opacity=1] {k};
        \draw[-] (B.north) -- (I_3.south) node [right, fill=white, opacity=.0, text opacity=1] {m};
        \draw[-] (B.east) -- (I_4.west) node [above, fill=white, opacity=.0, text opacity=1] {n};
        \draw[-] (C.south) -- (I_5.north) node [right, fill=white, opacity=.0, text opacity=1] {l};
    \end{tikzpicture}
    \caption{A tensor hypernetwork.}
    \label{fig:tensor:hypernetwork}
\end{figure}

\noindent
The contraction of the tensor hypernetwork $A \in \mathbb{R}^{I \times J \times K}, B \in 
\mathbb{R}^{K \times M \times N}$ and $C \in \mathbb{R}^{K \times L}$ in Figure 
\ref{fig:tensor:hypernetwork},

\[T_{ijmnl} = \sum_{k}A_{ijk} \cdot B_{kmn} \cdot C_{kl}\]
\noindent
\\
in modern Einstein notation written as

\[ijk,kmn,kl \rightarrow ijmnl\]
\noindent
\\
can be calculated in different orders. Either way, it is possible to get the same result by 
contracting $A$ and $B$ first, followed by $(AB) \cdot C$, by contracting $B$ and $C$ 
and then $A \cdot (BC)$ or by contracting $A$ and $C$, followed by $(AC) \cdot B$.
While the result of the contraction orders will be the same, the underlying number of 
operations may differ vastly. As a result, the order in which tensors are contracted
can drastically change the performance of an algorithm. We call this order the contraction
path.